\documentclass[a4paper,12pt]{article}   	% use "amsart" instead of "article" for AMSLaTeX format
\usepackage[papersize={216mm,330mm},tmargin=25mm,bmargin=25mm,lmargin=25mm,rmargin=25mm]{geometry}
\usepackage[english]{babel}
\usepackage[utf8]{inputenc}
%\usepackage{geometry}                		% See geometry.pdf to learn the layout options. There are lots.
%\geometry{letterpaper}                   		% ... or a4paper or a5paper or ... 
%\geometry{landscape}                		% Activate for rotated page geometry
%\usepackage[parfill]{parskip}    		% Activate to begin paragraphs with an empty line rather than an indent
\usepackage{graphicx}				% Use pdf, png, jpg, or eps§ with pdflatex; use eps in DVI mode
\usepackage{array}					% TeX will automatically convert eps --> pdf in pdflatex		
\usepackage{amssymb}
\usepackage{listings}
\usepackage{color}
\usepackage{xcolor}
\definecolor{dkgreen}{rgb}{0,0.6,0}
\definecolor{gray}{rgb}{0.5,0.5,0.5}
\definecolor{mauve}{rgb}{0.58,0,0.82}
\definecolor{beige}{HTML}{FFFFF0}

\lstset{frame=tb,
  language=Java,
  aboveskip=3mm,
  belowskip=3mm,
  showstringspaces=false,
  columns=flexible,
  basicstyle={\normalsize\ttfamily},
  numbers=none,
  numberstyle=\tiny\color{gray},
  keywordstyle=\color{blue},
  commentstyle=\color{dkgreen},
  stringstyle=\color{mauve},
  breaklines=true,
  breakatwhitespace=true,
  tabsize=4
}
\pagestyle{myheadings}


\linespread{1}
\title{Blinker Frequency}
\author{STM32F466RE}
\begin{document}
\maketitle

\subsection*{Default and Delayed Frequencies}
%Math
\large
If a firmware's only instruction is to blink an onboard LED,
blinks will occur at a natural [default core]
frequency ($f$) of 84MHz. We want to observe the blinking 
with natural vision, so we'll need to to add a number of cycle delays 
($N_d$) to our application to target a 
significantly-reduced, observable frequency ($f'$).\\[1 ex]

$$f = \frac{N}{T} = 84000000\ \frac{cycles}{second}\\[3 ex]$$

$$f' = \frac{N'}{T} = \frac{N/N_{d}}{T} \\[3 ex]$$
\large

\normalsize
*$f'$ is controllable, as opposed to $f$ which is determined by the chip's architecture.

\large
\begin{flushleft}
Initial implementation of cycle delays could be a 
$for\ loop$ that executes only one 'no-operation' instruction. At first 
glance, one iteration requires one cycle, so iterating ($i = 0;\ i < N;\ i++$)
predicts a requirement of $N_d = N_{loop}$ iterations, where $N_{loop} = N = 84\ million\ cycles$;
this corresponds to our desired $f'= 1\ Hz$.\\[2 ex] 

Consider the following code excerpt:\\[1 ex] 

%%%%%%% FOR LOOP CODE %%%%%%%%%%%
\begin{lstlisting}[backgroundcolor = \color{beige},
                    language = C,
                    xleftmargin = 2cm,
                    framexleftmargin = 1em]
  for (uint32_t i = 0; i < N; i++) {
        __asm__("nop");
  }
  gpio_toggle(led_port,led_pin);
  \end{lstlisting}
  %$$N_{loop} = \sum_{i=0}^{N}1 = 84000000\ iterations \stackrel{?}{=} 84000000\ cycles$$

Pushing this firmware and
observing the LED, it's clear that the order of magnitude is approximately correct, 
yet $f'$ is 4-5 times smaller than expected. Why is true 
if our application runs at 84MHz and our for loop executes only one $nop$ assembly instruction?\\[1 ex]

\end{flushleft}
\begin{itemize}
\setlength{\itemindent}{0.4in}
    \item The application does process on the system at $N$ cycles per second, 
    but each iteration of the $for\ loop$ costs more than 1 CPU cycle. 
    \item $N$ iterations of this $for\ loop$ ($N_{loops}$) costs $cN$ cycles ($c > 1$).
    \item $N_{loops} < N$ must be satisfied to achieve $f' \le 1.$
\end{itemize}

\begin{flushleft}

Error is generated solely from the following assumption:
$$N_{loop} = \sum_{i=0}^{N}1 = N = 84000000\ cycles$$
Empirically, it is clear that the above equality is incorrect and the following is true:
$$N_{loop} = \sum_{i=0}^{N}c_i > N$$
Since $N_{loop}$ is chosen by us, and $N$ is known, we can accurately 
determine the nunber of CPU cycles consumed by each iteration.\\[2 ex]

Re-arrange eq. 2 to obtain the number of cycles per $for\ loop$ iteration, denoted $c_i$:
$$N_{loop} = c_i\sum_{i=0}^{N} 1 = \frac{N}{f' * 1\ second}$$
$$c_i = \frac{1}{\sum_{i=0}^{N} 1} \frac{N}{f' * 1\ second}$$
$$c_i = \frac{1}{N} \frac{N}{f'*1\ second}$$
$$c_i = \frac{1}{f' * 1\ second}$$

Unsurprisingly, the observed frequency ($f'$) is inversely proportional to the number of cycles required for one iteration.
Our firmware over-estimated this requirement, resulting in the LED blinking $too\ slowly$. 
Reduce the number of iterations and the blinker rate will increase; reducing them by a factor of $c_i$ should yield our target of $f' = 1$.
That is, $f' = 1$ when $N_{loop} = \frac{1}{c_i}N$\\[2 ex]
%Another way to write this would be to isolate our target frequency, $f'$, to emphasize our goal of controlling the LED blink frequency.

Update the firmware and measure the blinker frequency. If the proposed relationship between $f'$ and $c_i$ is correct, 
$c_i$ will be an integer. Likeewise, machine instructions should not cost $partial$ cycles; computing operations are mostly discrete.\\[2 ex]

%%%%%%% FOR LOOP CODE %%%%%%%%%%%
\begin{lstlisting}[backgroundcolor = \color{beige}]
    uiint32_t Nloop = 84000000 / 4;
    for (uint32_t i = 0; i < Nloop; i++) {
          __asm__("nop");
    }
    gpio_toggle(led_port,led_pin);
    \end{lstlisting}

Re-assuring -- a blink frequency of $f' \approxeq 0.25\ cycles\ per\ second$. The inverse of this frequency gives the 
number of cycles required to complete one iteration of the $for\ loop$:

$$c_i \approxeq \frac{1}{0.25 \frac{cycles}{second}} \frac{1}{1\ second} = 4\ cycles$$

At 60fps, I measured $f' = 1.00 \frac{cycles}{second}$, suggesting our measurement of $c_i$ is highly accurate. 
This is probably not accurate down to the milisecond scale where error propogations are more significant.\\[2 ex]

Although inefficient, we determined the time (or cycles) required 
to execute simple machine instructions, without reading any corresponding assembly code.
%%Reduce the implemented delay by a factor of $c_i$ and re-measure $f'$ to confirm that it approaches $1\ \frac{cycle}{second}$.\\[2 ex]

%% INSERT 1 SECOND PHOTO %%


\end{flushleft}

%%%%%%%%%%%%%% PASTE CODE %%%%%%%%%%%%%%%%%%%%%
%%%%%%%%%%%%%% PASTE CODE %%%%%%%%%%%%%%%%%%%%%
%%%%%%%%%%%%%% PASTE CODE %%%%%%%%%%%%%%%%%%%%%
\subsection*{Firmware Source Code:}
\begin{lstlisting}[backgroundcolor = \color{beige}]
  // Blinker Firmware 2024.06.18
  #include <libopencm3/stm32/f4/rcc.h>
  #include <libopencm3/stm32/f4/gpio.h>
  
  #define LED_PORT (GPIOA)
  #define LED_PIN (GPIO5)
  
  static void rcc_setup(void) {
      rcc_clock_setup_pll(&rcc_hsi_configs[RCC_CLOCK_3V3_84MHZ]);
  }

  static void gpio_setup(void) {
      rcc_periph_clock_enable(RCC_GPIOA);
      gpio_mode_setup(LED_PORT,GPIO_MODE_OUTPUT,GPIO_PUPD_NONE,LED_PIN);
  }
  
  static void maintain_frequency_standard(uint32_t cycles) {
      for (uint32_t i = 0; i < cycles; i++) {
          __asm__("nop");
      }
  }
  
  int main(void) {
      rcc_setup();
      gpio_setup();
  
      while (1) {
          // Toggle the LED's state at freq (f')
          gpio_toggle(LED_PORT,LED_PIN);
          maintain_frequency_standard(84000000/4);
      }
      return 0;
  }


\end{lstlisting}

\end{document}